\chapter{Biologiczna potrzeba}
\label{section:biologiny_wstep}

\section{Bioinformatyka w praktyce}
\subsection{Medycyna}
Posiadamy aktualnie techniczne możliwości aby stosować indywidualną wiedzę o indywidualnym genomie człowieka do projektowania indywidualnego leczenia. Dzięki medycynie spersonalizowanej jesteśmy w stanie w wielu przypadkach przewidywać ryzyko zachorowania na daną chorobę. Analizując genom, możemy zidentyfikować mutacje, które powodują, że niektóre białka działają w sposób inny niż powinny. Wykrycie takich przypadków nie oznacza, że człowiek od razu zachoruje, ponieważ istnieje bardzo rozbudowana sieć zależności i jeżeli jedno białko nie działa to być może w ten sam sposób funkcjonuje inne. Ma to natomiast istotne znaczenie przy projektowaniu leczenia. W sytuacji gdy dane białko nie działa, a standardowo leczymy chorobę modyfikując to białko, nabywamy cenną informacje, że leczenie należy przeprowadzić w inny sposób. Dążymy do tego aby minimalizować koszt i czas leczenia pacjentów, oraz zwiększyć wykrywalność chorób.

\subsection{Farmaceutyka}
\subsection{Kryminalistyka}
\subsection{Sądownictwo}
\subsection{Rolnictwo}
\subsection{Archeologia}

\section{Bioinformatyka w ujęciu algorytmicznym}
\subsection{Bazy danych}
\subsection{big data}
\subsection{uczenie maszynowe}
\subsection{metody optymalizacji}
\subsection{teoria grafów}

\section{Problemy bioinformatyki}
Bioinformatyka, jak każda dziedzina naukowa boryka się z pewnymi problemami. Problem dotyczący baz danych polega na tym, że wiele baz powstaje i często potem nic się z nimi nie dzieje. Z każdym dniem poznajemy tysiące nowych sekwencji, które są deponowane w cyfrowych przechowalniach. Te podstawowe bazy są utrzymywane i z powodzeniem wykorzystywane do naukowych doświadczeń. Problemem są bazy wtórne, które wykorzystują podstawowe eksperymenty. Gromadzą one zbiór informacji pochodzących z różnych miejsc.

-niekontynuowane projekty \\
-dane bazują na eksperymentach, błędy propagujące się \\
-wiele różnych standardów, niekompatybilne \\
-nieinformatyczni biolodzy \\
-sposoby finansowania nauki \\

\section{Centralny dogmat bioinformatyki}
-dna, rna, białko \\
-informacja genetyczna \\
-struktura molekularna \\
-funkcja biochemiczna \\
-fenotyp

\section{Rozwój bioinformatyki}
\subsection{wcześniej}
\subsection{obecnie}
\subsection{w przyszłości}

\section{Genom ogórka}
-sggw
