\chapter{Biologiczna potrzeba}
\label{section:biologiny_wstep}

\section{Bioinformatyka w praktyce}
\subsection{medycyna}
\subsection{farmaceutyka}
\subsection{kryminalistyka}
\subsection{sądownictwo}
\subsection{rolnictwo}
\subsection{archeologia}

\section{Bioinformatyka w ujęciu algorytmicznym}
\subsection{bazy danych}
\subsection{big data}
\subsection{uczenie maszynowe}
\subsection{metody optymalizacji}
\subsection{teoria grafów}

\section{Problemy bioinformatyki}
-niekontynuowane projekty \\
-dane bazują na eksperymentach, błędy propagujące się \\
-wiele różnych standardów, niekompatybilne \\
-nieinformatyczni biolodzy \\
-sposoby finansowania nauki \\

\section{Centralny dogmat bioinformatyki}
-dna, rna, białko \\
-informacja genetyczna \\
-struktura molekularna \\
-funkcja biochemiczna \\
-fenotyp

\section{Rozwój bioinformatyki}
\subsection{wcześniej}
\subsection{obecnie}
\subsection{w przyszłości}

\section{Genom ogórka}
-sggw
