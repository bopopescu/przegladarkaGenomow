
\chapter{Wstęp}
\label{section:wstep}

Obecnie żyjemy w epoce dynamicznego rozwoju technologicznego i nauk przyrodniczych. Spowodowało to rozwój i powstanie nowych dziedzin naukowych a wśród nich bioinformatyki. Z pomocą metod matematycznych i informatycznych jest wykorzystywana do przetwarzania ogromnych zbiorów danych uzyskiwanych podczas badań naukowych. Konieczność przechowywania, sprawnego przeglądania, udostępniania pozyskanych danych zrodziła potrzebę powstawania narzędzi zwanymi przeglądarkami genomów.
% https://www.kierunki.net/bioinformatyka

\section{Cel i zakres pracy}
\label{section:cel_i_zakres_pracy}
W celu zsekwencjonowania genomu ogórka, w 2008r. zostało powołane Polskie Konsorcjum Sekwencjonowania Genomu Jądrowego Ogórka. Znaczna część jego członków to pracownicy Katedry Genetyki, Hodowli i Biotechnologii Roślin Szkoły Głównej Gospodarstwa Wiejskiego w Warszawie. Praca jest naturalną odpowiedzią na zapotrzebowanie na nowe narzędzia niezbędne do wygodniejszego i efektywniejszego  wykonywania badań nad otrzymanym genomem. Aplikacja została wykonana we współpracy z pracownikami SGGW. Jednym z głównych założeń jest to, aby przeglądarka była użyteczna dla innych zespółów naukowych badających genomy różnych organizmów.

\subsection{Hipotezy}

\begin{enumerate}[I.]
	\item \textit{
		Można zaproponować strukturę bazy danych do przechowywania informacji opisujących genomy, taką że będzie ona elastyczna, łatwa w modyfikacji i~uniezależniona od semantyki przechowywanych struktur biologicznych.
		} \\
	%\todo{krótki opis - duże dane, często nieznana struktura docelowa, wiele standardów, zanieczyszczone dane}
	\break
	Dane opisujące genomy są mocno zróżnicowane pod względem struktury w zależności od ich znaczenia biologicznego. Bazy danych projektowane pod kątem przechowywania genomu konkretnego organizmu bądź ograniczonego zbioru typów sekwencji dość mocno specjalizują aplikację i ograniczają lub utrudniają stosowanie jej w bardziej ogólnych zastosowaniach. Jednym z celów pracy jest próba rozwiązania tego problemu poprzez zaproponowanie elastycznej struktury bazodanowej oraz weryfikacja jej przez import rzeczywistych danych.\\
	
	\item \textit{
		Można dostarczyć abstrakcję widoków na dane genetyczne, umożliwiające analizę danych z różnych perspektyw w kontekście licznych zbiorów
		danych genetycznych.
		} \\
	%\todo{różne podłoże semantyczne sekwencji, często sprzeczne niepełne informację, potrzeba różnych perspektyw na ten sam chromosom }
	\break
	Informacje genetyczne pozyskiwane przez badaczy są na początkowych stadiach często niespójne, niekompletne i zanieczyszczone. Składanie sekwencji nadrzędnych z podrzędnych można zrobić na wiele sposobów. Celem jest dostarczenie mechanizmu umożliwiającego użytkownikowi patrzenie na ten sam genom z różnych perspektyw poprzez priorytetyzację wybranych typów podrzędnych sekwencji co nada analizowanym danym wielowymiarowości.\\
	
	\item \textit{
		Można dostarczyć aplikację do przechowywania i analizy genomów, która w~przystępny sposób umożliwi przeglądanie sekwencji genetyczych bez konieczności posiadania wydajnej maszyny klienckiej. 
		} \\
		\break
	%todo
	Możność szybkiego i prezycyjnego przeglądania obszernych genomów o zróżnicowanej strukturze wraz ze sposobnością ich analizy przy wymaganiach zachowania płynności aplikacji stawia wyzwanie architektoniczne.
	
	Szybkie i precyzyjne przeglądanie genomu jest bardzo porządaną funkcjonalnością przez biologów w ich codziennej pracy. Pnalizy obszernych genomów w połączeniu
	Celem jest stworzenie aplikacji pozwalającej na interaktywne wyświetlanie genomów i adnotacji w makro i mikro 
	\todo{duże dane, muszą być wydajne algorytmy, aplikacja lekka, intuicyjna, z dobrym api do dzielenia się z innymi zespołami}
	
\end{enumerate}

\subsection{Plan badań}
- generyczna baza danych \\
-genom ogorka, analiza danych sggw, produkcja chromosomów \\
-możliwość wdrożenia wydajnych implementacji algorytmów \\
-wygoda uzytkownika \\
-lekkie przeglądanie w skali makro i mikro \\

\section{Układ pracy}
\label{section:uklad_pracy}

\section{Przegląd literatury}
\label{section:przeglad_literatury}
