
\chapter{Wstęp}
\label{section:wstep}

Obecnie żyjemy w epoce dynamicznego rozwoju technologicznego i nauk przyrodniczych. Spowodowało to rozwój i powstanie nowych dziedzin naukowych a wśród nich bioinformatyki. Z pomocą metod matematycznych i informatycznych jest ona wykorzystywana do przetwarzania ogromnych zbiorów danych uzyskiwanych podczas badań naukowych. 

Przykładem obszaru, który zostanie wkrótce zmieniony przez bioinformatykę jest medycyna.
Posiadamy aktualnie techniczne możliwości aby stosować indywidualną wiedzę o indywidualnym genomie człowieka do projektowania indywidualnego leczenia.
Takie podejście nazywamy medycyną spersonalizowaną.
Dzięki niej, jesteśmy w stanie w wielu przypadkach przewidywać ryzyko zachorowania na daną chorobę.
Analizując genom, możemy zidentyfikować mutacje, które powodują, że niektóre białka działają w sposób inny niż powinny.
Wykrycie takich przypadków nie oznacza, że człowiek od razu zachoruje, ponieważ istnieje bardzo rozbudowana sieć zależności i jeżeli jedno białko nie działa to być może w ten sam sposób funkcjonuje inne.
Ma to natomiast istotne znaczenie przy projektowaniu leczenia.
W sytuacji gdy dane białko nie działa, a standardowo leczymy chorobę modyfikując to białko, nabywamy cenną informacje, że leczenie należy przeprowadzić w inny sposób.
Dążymy do tego aby minimalizować koszt i czas leczenia pacjentów, oraz zwiększyć wykrywalność chorób.

Konieczność przechowywania, sprawnego przeglądania, udostępniania pozyskanych danych zrodziła potrzebę powstawania narzędzi zwanymi przeglądarkami genomów.
% https://www.kierunki.net/bioinformatyka

\section{Cel i zakres pracy}
\label{section:cel_i_zakres_pracy}
W celu zsekwencjonowania genomu ogórka, w 2008r. zostało powołane Polskie Konsorcjum Sekwencjonowania Genomu Jądrowego Ogórka.
Znaczna część jego członków to pracownicy Katedry Genetyki, Hodowli i Biotechnologii Roślin Szkoły Głównej Gospodarstwa Wiejskiego w Warszawie.
Praca jest naturalną odpowiedzią na zapotrzebowanie na nowe narzędzia niezbędne do wygodniejszego i efektywniejszego  wykonywania badań nad otrzymanym genomem.
Aplikacja została wykonana we współpracy z pracownikami SGGW.
Jednym z głównych założeń jest to, aby przeglądarka była użyteczna również dla innych zespołów naukowych badających genomy różnych organizmów.
W trakcie realizacji pracy będę chciał udowodnić następujące hipotezy:

%\subsection*{Hipotezy}

\begin{enumerate}[I.]
	\item \textit{
		Można zaproponować strukturę bazy danych do przechowywania informacji opisujących genomy, taką że będzie ona elastyczna, łatwa w modyfikacji i~uniezależniona od semantyki przechowywanych struktur biologicznych.
		} \\
	%\todo{krótki opis - duże dane, często nieznana struktura docelowa, wiele standardów, zanieczyszczone dane}
	\break
	Dane opisujące genomy są mocno zróżnicowane pod względem struktury w zależności od ich znaczenia biologicznego. Bazy danych projektowane pod kątem przechowywania genomu konkretnego organizmu bądź ograniczonego zbioru typów sekwencji dość mocno specjalizują aplikację i ograniczają lub utrudniają stosowanie jej w bardziej ogólnych zastosowaniach. Jednym z celów pracy jest próba rozwiązania tego problemu poprzez zaproponowanie elastycznej struktury bazodanowej oraz weryfikacja jej przez import rzeczywistych danych.\\
	
	\item \textit{
		Można dostarczyć abstrakcję widoków na dane genetyczne, umożliwiające analizę danych z różnych perspektyw w kontekście licznych zbiorów
		danych genetycznych.
		} \\
	%\todo{różne podłoże semantyczne sekwencji, często sprzeczne niepełne informację, potrzeba różnych perspektyw na ten sam chromosom }
	\break
	Informacje genetyczne pozyskiwane przez badaczy są na początkowych stadiach często niespójne, niekompletne i zanieczyszczone. 
	Składanie sekwencji nadrzędnych z podrzędnych można zrobić na wiele sposobów. 
	\przeredagowac{Celem jest dostarczenie mechanizmu umożliwiającego użytkownikowi patrzenie na ten sam genom z różnych perspektyw poprzez priorytetyzację wybranych typów podrzędnych sekwencji co nada analizowanym danym wielowymiarowości.}
	
	\break
	\item \textit{
		Można dostarczyć aplikację do przechowywania i analizy genomów, która w~przystępny sposób umożliwi przeglądanie sekwencji genetycznych bez konieczności posiadania wydajnej maszyny klienckiej. 
		} \\
		\break
	%todo
	Możność szybkiego i precyzyjnego przeglądania obszernych genomów o zróżnicowanej strukturze wraz ze sposobnością ich analizy przy wymaganiach zachowania płynności działania aplikacji stawia z perspektywy informatyki wyzwanie architektoniczne. Celem jest stworzenie przenośnej aplikacji pozwalającej na interaktywne wyświetlanie genomów i adnotacji w skali makro i mikro wraz z elementami służącymi do poszukiwania sekwencji podobnych.
	
\end{enumerate}

\subsection*{Plan badań}
\todo{todo - po opisaniu badań na koniec tu wrócić\\}
\todo{
- generyczna baza danych \\
-genom ogorka, analiza danych sggw, produkcja chromosomów \\
-możliwość wdrożenia wydajnych implementacji algorytmów \\
-wygoda uzytkownika \\
-lekkie przeglądanie w skali makro i mikro \\
}

\section{Układ pracy}
\label{section:uklad_pracy}
\todo{todo}

% przeniosione do rozdziału "sekwencjonowanie i adnotacja genomu"
%\section{Przegląd literatury}
%\label{section:przeglad_literatury}
