\pdfoutput=1
\pdfcompresslevel=9
\pdfinfo
{
    /Author (Autor)
    /Title (Tytuł pracy magisterskiej)
    /Subject (Bioinformatyka)
    /Keywords (DNAASM)
}
%\documentclass[a4paper,polish,onecolumn,oneside,floatssmall,11pt,titleauthor,wide,openright]{mwrep}
%\usepackage[scale={0.7,0.8},paper=a4paper,twoside]{geometry}
\documentclass[a4paper,onecolumn,twoside,openright,11pt,wide,floatssmall]{mwrep}
% \usepackage{polish}
\usepackage{amsmath}
\usepackage{amsfonts}
\usepackage{amssymb}
\usepackage{amsthm}
\usepackage{bookman}
\usepackage{algorithm}
\usepackage{algpseudocode}
\usepackage{geometry}
\usepackage{placeins}
%\usepackage{slashbox}
\usepackage[utf8x]{inputenc}
\usepackage[T1]{fontenc}
\usepackage{footnote}
% \usepackage{fancyvrb}
% \usepackage{t1enc}
% \usepackage[pdftex, bookmarks]{hyperref}
\usepackage[pdftex, bookmarks=false]{hyperref}

\floatname{algorithm}{Algorytm}
\renewcommand{\algorithmicforall}{\textbf{for each}}

\usepackage{caption}
%\usepackage[vlined]{algorithm2e}

\def\url#1{{ \tt #1}}

\usepackage{listings}








\usepackage{fancyhdr}
\pagestyle{fancy}
\fancyhf{}
\fancyhead[CE,CO]{\leftmark}
\fancyfoot[LE,RO]{\thepage}
\usepackage[toc,page]{appendix}

\fancypagestyle{plain}{%
\fancyhf{}%
\fancyfoot[LE,RO]{\thepage}}







% marginesy
\textwidth\paperwidth
\advance\textwidth -55mm
\oddsidemargin-0.9in
\advance\oddsidemargin33mm
\evensidemargin0.9in
\advance\evensidemargin-33mm

\topmargin -1in
\advance\topmargin 25mm
\setlength\textheight{48\baselineskip}
\addtolength\textheight{\topskip}
\marginparwidth15mm

\clubpenalty=10000 % to kara za sierotki
\widowpenalty=10000 % nie pozostawia wdów
\brokenpenalty=10000 % nie dzieli wyrazów pomiędzy stronami
\sloppy

\tolerance4500
\pretolerance250
\hfuzz=1.5pt
\hbadness1450

\linespread{1.5} %interlinia
% ŻYWA PAGINA
\renewcommand{\chaptermark}[1]{\markboth{\scshape\small\bfseries \
#1}{\small\bfseries \ #1}}
\renewcommand{\sectionmark}[1]{\markboth{\scshape\small\bfseries\thesection.\
#1}{\small\bfseries\thesection.\ #1}}
\renewcommand{\headrulewidth}{0.5pt}
\renewcommand{\footrulewidth}{0.pt}
%%\pagestyle{uheadings}

\usepackage[pdftex]{color,graphicx}
\usepackage[polish]{babel}

% \textheight232mm
% \setlength{\textwidth}{\textwidth}
% \setlength{\oddsidemargin}{\evensidemargin}
% \setlength{\evensidemargin}{0.3cm}
\usepackage[sort, compress]{cite}

%\usepackage{multibib}
%\newcites{bk,st,doc,web}{Książki i~artykuły,Standardy i~zalecenia,Dokumentacja produktów,Publikacje i~serwisy internetowe}

\theoremstyle{definition}
\newtheorem{defn}{Definicja}[chapter]
\newtheorem{conj}{Teza}[section]
\newtheorem{conjmain}{Teza}
\newtheorem{exmp}{Przykład}[chapter]

\theoremstyle{plain}% default
\newtheorem{thm}{Twierdzenie}[section]
\newtheorem{lem}[thm]{Lemat}
\newtheorem{prop}[thm]{Hipoteza}
\newtheorem*{cor}{Wniosek}

\theoremstyle{remark}
\newtheorem*{rem}{Uwaga}
\newtheorem*{note}{Uwaga}
\newtheorem{case}{Przypadek}[chapter]

\definecolor{ListingBackground}{rgb}{0.95,0.95,0.95}

\begin{document}

% kody źródłowe wplatane w tekst
\lstdefinestyle{incode}
{
basicstyle={\footnotesize},
keywordstyle={\bf\footnotesize\color{blue}},
commentstyle={\em\footnotesize\color{magenta}},
numbers=left,
stepnumber=5,
firstnumber=1,
numberfirstline=true,
numberblanklines=true,
numberstyle={\sf\tiny},
numbersep=10pt,
tabsize=2,
xleftmargin=17pt,
framexleftmargin=3pt,
framexbottommargin=2pt,
framextopmargin=2pt,
framexrightmargin=0pt,
showstringspaces=true,
backgroundcolor={\color{ListingBackground}},
extendedchars=true,
% title=\lstname,
captionpos=b,
% abovecaptionskip=1pt,
% belowcaptionskip=1pt,
frame=tb,
framerule=0pt,
}

% kody źródłowe z podpisem
\lstdefinestyle{outcode}
{
basicstyle={\footnotesize},
keywordstyle={\bf\footnotesize\color{blue}},
commentstyle={\em\footnotesize\color{magenta}},
numbers=left,
stepnumber=5,
firstnumber=1,
numberfirstline=true,
numberblanklines=true,
numberstyle={\sf\tiny},
numbersep=10pt,
tabsize=2,
xleftmargin=17pt,
framexleftmargin=3pt,
framexbottommargin=2pt,
framextopmargin=2pt,
framexrightmargin=0pt,
showstringspaces=true,
backgroundcolor={\color{ListingBackground}},
extendedchars=true,
% title=\lstname,
captionpos=b,
% abovecaptionskip=1pt,
% belowcaptionskip=1pt,
frame=tb,
framerule=0.1pt,
}

%\counterwithin{exmp}{chapter}
%\counterwithout{exmp}{section}
\renewcommand*\lstlistingname{Wydruk}
\renewcommand*\lstlistlistingname{Spis wydruków}

\pagenumbering{roman}
\renewcommand{\baselinestretch}{1.0}
\raggedbottom

\begin{titlepage}
    % Strona tytułowa


    % Życiorys
    \newpage\thispagestyle{empty}
    \begin{tabular}{p{5cm} p{12cm}}
    \begin{minipage}{5cm}
    \center
    %% \includegraphics[height=6.5cm,width=4.5cm]{img/ja.png}
    \end{minipage}
    &
    \begin{minipage}{12cm}
    \begin{flushleft}
    \par\noindent\vspace{0\baselineskip}
    \begin{tabular}[h]{l l}
    {\normalsize\it Specjalność:} & Informatyka -- \\
    & Systemy Informacyjno Decyzyjne \\
    \end{tabular}
    \par\noindent\vspace{1\baselineskip}
    \begin{tabular}[h]{l l}
    {\normalsize\it Data urodzenia:} & {\normalsize 03.04.1992}
    \end{tabular}
    \par\noindent\vspace{1\baselineskip}
    \begin{tabular}[h]{l l}
    {\normalsize\it Data rozpoczęcia studiów:} & {\normalsize 20.02.2016}%1 października 2012 r.}
    \end{tabular}
    \par\noindent\vspace{1\baselineskip}
    \end{flushleft}
    \end{minipage}
    \end{tabular}
    \vspace*{0\baselineskip}
    \begin{center}
	{\large\bfseries Życiorys}\par\bigskip
    \end{center}
	\mockup{
	Urodziłem się 3 kwietnia 1992 roku w~Kielcach. W~2008 roku rozpocząłem naukę w~II Liceum Ogólnokształcącym im. Jana Śniadeckiego w~Kielcach, w~klasie o~podstawie programowej w~zakresie rozszerzonym z~matematyki, informatyki, fizyki i~astronomii . W~2011 roku uzyskałem wykształcenie średnie oraz świadectwo dojrzałości... 
	}
    \indent

    \par
    \vspace{2\baselineskip}
    \hfill\parbox{15em}{{\small\dotfill}\\[-.3ex]
    \centerline{\footnotesize podpis studenta}}\par
    \vspace{1\baselineskip}
    \begin{center}
 	{\large\bfseries Egzamin dyplomowy} \par\bigskip\bigskip
    \end{center}
    \par\noindent\vspace{1.0\baselineskip}
    Złożył egzamin dyplomowy w dn. \dotfill
    \par\noindent\vspace{1.0\baselineskip}
    Z wynikiem \dotfill
    \par\noindent\vspace{1.0\baselineskip}
    Ogólny wynik studiów \dotfill
    \par\noindent\vspace{1.0\baselineskip}
    Dodatkowe wnioski i uwagi Komisji \dotfill
    \par\noindent\vspace{1.0\baselineskip}
    \dotfill

    % Streszczenie po polsku
    \newpage\thispagestyle{empty}
    \vspace*{2\baselineskip}
    \begin{center}
	{\large\bfseries Streszczenie}\par\bigskip
	\vspace*{2\baselineskip}
    \end{center}

    \itshape
    \todo{todo}
    \vspace*{3\baselineskip}

    \noindent{\bf Słowa kluczowe}: {\itshape \todo{todo.}}
    %\par
    %\vspace{4\baselineskip}
    
    % Streszczenie po angielsku
    \newpage\thispagestyle{empty}
    \vspace*{2\baselineskip}
    \begin{center}
	{\large\bfseries Abstract}\par\bigskip
	\vspace*{2\baselineskip}
    \end{center}
    \noindent{\bf Title}: {\itshape \todo{XXX}}\par
    \vspace*{1\baselineskip}
    \itshape
    \todo{XXX}
    \vspace*{3\baselineskip}

    \noindent{\bf Key words}: {\itshape \todo{XXX}.}

\end{titlepage}

% ex: set tabstop=4 shiftwidth=4 softtabstop=4 noexpandtab fileformat=unix filetype=tex spelllang=pl,en spell:


%\newpage\null\thispagestyle{empty}\newpage

% Podziękowania
\newpage\thispagestyle{empty}


\tableofcontents
% \addcontentsline{toc}{chapter}{{Przedmowa1}{vii}}{vii}

% \chapter*{Spis tablic, rysunków i~wydruków}
% \listoftables
% \listoffigures
% \listofalgorithms
% \lstlistoflistings

%\setlength{\baselineskip}{7mm}
\newpage
\pagenumbering{arabic}
\raggedbottom{}
\setcounter{page}{1}




\chapter{Wstęp}
\label{section:entry}

\section{Cel i zakres pracy}
\label{section:cel_i_zakres_pracy}

\section{Układ pracy}


\input{theory}

\chapter{Opis algorytmów}
\label{ch:algorithms}



\chapter{Implementacja}
\label{design}



\chapter{Badania}
\label{section:badania}



\chapter{Podsumowanie}
\label{summary}



%\chapter*{Bibliografia}
\nocite{*}
\bibliographystyle{plplain}
%\bibliographystylebk{plplain}
%\bibliographystylest{plplain}
%\bibliographystyledoc{plplain}
% \bibliographystyleweb{plplain}
%\bibliographybk{BIB/books}
%\bibliographyst{BIB/books}
%\bibliographydoc{BIB/books}
% \bibliographyweb{BIB/books}

\bibliography{bib/mgr}

\appendix

%% \input{spis_rysunkow}

%% \input{spis_tabel}

%\input {spis_zalacznikow}

\chapter{Instrukcje użytkownika}
\label{instructions}



\end{document}

% ex: set tabstop=4 shiftwidth=4 softtabstop=4 noexpandtab fileformat=unix filetype=tex spelllang=pl,en spell:

